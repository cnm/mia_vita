\documentclass[10pt,a4paper,oneside]{book}

\usepackage[T1]{fontenc}
\usepackage[portuguese]{babel}
\usepackage[utf8]{inputenc}
\usepackage{amsmath}
\usepackage{color}
\usepackage{amsfonts}
\usepackage[pdfborder={0 0 0},
	    plainpages=false,
	    pdfpagelabels,
	    pdftex,
 	    dvipdfmx,
            pdfauthor={Frederico Mendonça Passos de Lopes e Gonçalves},
            pdftitle={Miavita Developed Software User Manual},
            pdfsubject={Miavita User Manual},
            pdfproducer={pdflatex with hyperref},
            pdfcreator={pdflatex}]{hyperref}
\usepackage{simplemargins}
\usepackage{graphicx}
\usepackage{listings}

\setallmargins{1in}

\title{\textbf{Manual de Software Desenvolvido para o Miavita}}
\author{Frederico Gonçalves\\MEIC-T, 57843\\\href{mailto:frederico.lopes.goncalves@gmail.com}{frederico.lopes.goncalves@gmail.com}}
\date{\today}

\begin{document}
\maketitle

\tableofcontents

\chapter{Preparação do Sistema}

\section{Cross-compilation}

Esta secção demonstra como compilar programas para o ARM a partir de uma outra plataforma (cross-compiling). As vantagens são óbvias. O processo de compilação é muito mais rápido e por vezes não é de todo possível compilar código no ARM, como é o caso dos módulos para o kernel. Isto deve-se ao facto de a \emph{source} do kernel não estar presente no ARM, pois é demasiado grande.

De modo a poder compilar código para o ARM noutra plataforma é necessário uma versão do \emph{gcc} preparada para tal. Esta pode ser obtida no site \emph{ftp} da TS - \href{ftp://ftp.embeddedarm.com/ts-arm-sbc/ts-7500-linux/cross-toolchains/}{ftp://ftp.embeddedarm.com/ts-arm-sbc/ts-7500-linux/cross-toolchains/}. O pacote não traz apenas uma versão diferente do \emph{gcc}, traz também uma versão diferente da \emph{libc}, denominada \emph{uclibc} (micro libc) e ainda um conjunto de ferramentas associadas ao processo de compilação. A \emph{uclibc} destingue-se da \emph{libc} por ser muito mais pequena e estar desenhada para correr em sistemas embebidos. É perfeitamente normal que funções presentes na \emph{libc} não se encontrem na \emph{uclibc}. Por esta razão é extremamente recomendável usar a \emph{cross-toolchains} mais recente possível, de modo a trazer a versão mais actual da \emph{uclibc}.

Uma vez feito o download basta desarquivar o ficheiro para um local onde este fique permanentemente. Por exemplo:

\begin{center}
{\tt tar xvzf crosstool-linux-gcc.tar.gz  -C $\sim$}
\end{center}

Os binários para compilar programas encontram-se em:

\begin{center}
{\tt $\sim$/arm-unknown-linux-gnu/bin/}
\end{center}

Embora seja possível usar o compilador especificando o caminho todo, é bastante mais produtivo adicioná-lo à \emph{PATH}, de modo a poder ser chamado de qualquer lado. Para tornar o processo automático é também recomendável inserir o próximo comando no ficheiro {\tt $\sim$/.bashrc}.

\begin{center}
{\tt export PATH=\$PATH:/home/fred/arm-unknown-linux-gnu/bin/}
\end{center}

Se tudo correr bem, o seguinte comando deve produzir a versão do \emph{gcc} para o ARM.

\begin{center}
{\tt arm-unknown-linux-gnu-gcc --version}
\end{center}

Estes são todos os passos necessários para preparar o sistema para compilar programas para o ARM. Para usar o compilador basta fazê-lo da mesma forma que usaríamos o \emph{gcc}. Como por exemplo:

\begin{center}
{\tt arm-unknown-linux-gnu-gcc -o exec main.c}
\end{center}

Para compilar o kernel também é necessário usar este novo compilador, mas neste caso é necessário alterar a Makefile do kernel e substituir a variável \emph{CROSS\_COMPILE} por \emph{CROSS\_COMPILE=arm-unknown-linux-gnu-} (Note-se que não é suposto ter \emph{gcc} no final).

\section{Código do Projecto}

Esta secção explica como obter o código do projecto. Todo o código desenvolvido para o projecto encontra-se no \emph{github}, precisamente em \href{https://github.com/cnm/mia\_vita}{https://github.com/cnm/mia\_vita}. 

Uma vez feito o \emph{clone} do repositório é possível observar as seguintes directorias principais:

\begin{flushleft}
  {\tt
    kernel\_sender/\\
    interruption/\\
    rt2501/\\
    rt3070/\\
    fred\_framework/
  }
\end{flushleft}

Estas directorias são as que contém a maior parte do código desenvolvido e/ou modificado e é nelas que o resto do manual se foca.

De notar que embora o comando {\tt git branch -a} mostre vários \emph{branches} remotos, o código principal encontra-se no \emph{master}.

\chapter{Código}

 Esta secção explica todo o código desenvolvido. Mais precisamente, descreve quais as funções de cada componente, como compilá-los e correr.

  \section{Kernel Sender e Programa Servidor}
  
  A directoria {\tt kernel\_sender/} contém o código que prepara os pacotes e envia-os para o nó \emph{sink}. Este por sua vez deverá ter uma instância do programa que recebe os dados a correr. Este programa encontra-se em {\tt kernel\_sender/user}.

  \subsubsection{Kernel Sender}\label{sec:sender}

    O programa que envia os dados (\emph{sender\_kthread.ko}) opera em \emph{kernel land}. A principal razão para este facto prende-se com a forma como as amostras são recolhidas do ADC. Estas são recolhidas por um módulo de kernel (\emph{int\_mod.ko} localizado em {\tt interruption/}) criado pelo João Trindade que preenche um \emph{buffer} com tais amostras e as temporiza. Seria possível ler este \emph{buffer} a partir de um programa em \emph{user land}, contudo isto traria um \emph{overhead} indesejável, pelo que optou-se por fazer o programa cliente também em \emph{kernel land}.

    Como seria de esperar, o módulo que envia os dados depende do módulo que recolhe as amostras. Como tal, este último tem de ser compilado primeiro. Na directoria {\tt kernel\_sender/} encontra-se um \emph{script} com o nome \emph{compile.bash}. Este \emph{script} trata de compilar tanto o módulo \emph{int\_mod.ko} como o módulo \emph{sender\_kthread.ko}, resolvendo todas as dependências.

    Quando o módulo \emph{sender\_kthread.ko} foi desenvolvido, o protocol de sincronização estava em fase de testes. O objectivo dos testes era comparar o atraso dado por dispositivos GPS, face ao atraso medido pelo protocolo. Por esta razão, várias zonas do código encontram-se circunscritas por {\tt \#ifdef \_\_GPS\_\_ ... \#endif}. Para que o código seja compilado para os testes com GPS basta passar ao compilador a \emph{flag} {\tt -D\_\_GPS\_\_}. Como isto trata-se de um módulo de kernel, é preciso abrir a Makefile e adicionar a linha:

    \begin{center}
      {\tt EXTRA\_CFLAGS+=-D\_\_GPS\_\_}
    \end{center}

    Note-se que ao compilar este módulo para correr os testes com GPS, é necessário também compilar o código do driver (Secção \ref{sec:driver}) e o programa que recebe os dados com a mesma \emph{flag}.

    É extremamente importante perceber que esta \emph{flag} não faz com que o valor de criação do pacote passe a ser medido pelo GPS. Apenas prepara o código para os testes com GPS. O valor de criação do pacote continua a ser medido pelo protocolo de sincronização. Para alterar o código de modo a usar o valor do GPS como tempo de criação do pacote é necessário alterar as zonas marcadas com comentários no código.
    (TODO Fred Marca com comentários e faz listagem dos ficheiros sff)

    É ainda possível compilar o módulo com informação de \emph{debug} que é emitida para {\tt /var/log/syslog}. Neste caso é necessária a \emph{flag} {\tt -DDBG}.

    O módulo \emph{sender\_kthread.ko} tem uma funcionalidade muito simples. Limita-se a criar uma \emph{thread}, que por sua vez abre um socket e de X em X tempo vai lendo N amostras do \emph{buffer} de amostras em \emph{int\_mod.ko} (ficheiro source: proc\_entry.c). Por cada amostra é criado um pacote e os dados são enviados para o \emph{sink}. É preciso especificar pelo menos o IP do \emph{sink} e o do próprio nó. Isto pode ser feito em \emph{runtime}. Para saber todos os parâmetros de um módulo basta usar o comando:

    \begin{flushleft}
      {\tt 
        modinfo sender\_kthread.ko\\\hfill\\
        filename:       ./sender\_kthread.ko\\
        description:    This module spawns a thread which reads the buffer exported by João and
                        sends samples accross the network.\\
        author:         Frederico Gonçalves, [frederico.lopes.goncalves@gmail.com]\\
        license:        GPL v2\\
        depends:        int\_mod\\
        vermagic:       2.6.24.4 mod\_unload ARMv4 \\
        parm:           bind\_ip:This is the ip which the kernel thread will bind to. Default is localhost. (charp)\\
        parm:           sink\_ip:This is the sink ip. Default is localhost. (charp)\\
        parm:           sport:This is the UDP port which the sender thread will bind to. Default is 57843. (ushort)\\
        parm:           sink\_port:This is the sink UDP port. Default is 57843. (ushort)\\
        parm:           node\_id:This is the identifier of the node running this thread. Defaults to 0. (ushort)\\
        parm:           read\_t:The sleep time for reading the buffer. (uint)
      }
    \end{flushleft}

    Especificar parametros para um módulo é bastante simples. Como exemplo:

    \begin{flushleft}
      {\tt insmod sender\_kthread.ko bind\_ip=''192.168.2.123'' sink\_ip=''192.168.2.1''}
    \end{flushleft}

    Por fim é preciso ter em conta a especificação do pacote de dados (Secção \ref{sec:packet_specification}). Todos os campos são enviados em \emph{network byte order}, que é \emph{Big Endian}. Os processadores ARM podem funcionar tanto em \emph{Little Endian}, como em \emph{Big Endian}. Infelizmente, os processadores das placas TS-7500 funcionam em \emph{Little Endian}, pelo que os dados têm de ser convertidos antes de serem enviados. Para tipos de dados alinhados, isto é, inteiros de 16, 32 e 64 bits o kernel já fornece funções que fazem a conversão. Contudo, cada amostra tem 24 bits, pelo que não existe nenhuma função que faça a conversão por nós. Esta foi então implementada na função \emph{read\_nsamples} localizada no ficheiro \emph{interruption/proc\_entry.c}. O modo como foi implementada foi pensado para ser o mais rápido possível, evitando ciclos. Contudo é preciso ter especial cuidado com o seguinte. Esta conversão depende de dois grande factores:

    \begin{enumerate}
    \item Assume que o ARM trata os dados como \emph{Little Endian}. Se por alguma razão o \emph{hardware} mudar, é preciso verificar se esta conversão está a ser feita correctamente.
    \item Assume que o código no ficheiro \emph{interruption/fpga.c} lê as amostras de uma forma especifica. Se este código mudar, é necessário verificar se a conversão contínua a ser bem feita. Por outras palavras, o código da função \emph{read\_nsamples} é extremamente dependente do código do ficheiro \emph{interruption/fpga.c}.
    \end{enumerate}

    \subsubsection{Programa Servidor}
    
    Na directoria {\tt kernel\_sender/user} encontra-se o programa que recebe os dados do módulo {\tt sender\_kthread.ko}. Para compilar o programa basta usar a Makefile dentro da directoria. A Makefile define a variável {\tt CC}, que é usada para determinar qual o compilador a usar. Para compilar o programa para o ARM, basta alterar esta variável para reflectir o caminho para o \emph{cross-compiler}. Por exemplo:

    \begin{flushleft}
      {\tt make CC=arm-unknown-linux-gnu-gcc}
    \end{flushleft}

    Tal como os módulos do kernel, este código também pode ser compilado para ser usado nos testes com GPS. Neste caso é preciso adicionar à \emph{flag} {\tt CFLAGS} a opção {\tt -D\_\_GPS\_\_}, dentro da Makefile. O \emph{output} do programa são dois ficheiros, um em formato binário e outro em formato json (ver Secção \ref{sec:packet_specification}). É preciso ter em conta que quando são feitos os testes com GPS, o campo \emph{timestamp} reflecte o atraso medido pelo protocolo de sincronização e um campo adicional (\emph{gps\_us}) reflecte o atraso medido pelo GPS. O primeiro encontra-se em nanosegundos e o último em microsegundos.

    Os pacotes são escritos para o ficheiro binário tal e qual como chegam ao socket. Contudo, antes de serem escritos para o formato json, são convertidos para a \emph{byte order} do CPU. Tanto o ficheiro binário, como o ficheiro json são rotativos. Os pacotes são escritos para um ficheiro A. Na próxima escrita, o ficheiro A é movido para um ficheiro B e rescrito com os novos pacotes. Todos os nomes e tamanhos do ficheiro são configuráveis em \emph{runtime}:

    \begin{flushleft}
      ./main -h\\\hfill\\
      Usage: ./main [-i <interface>] [-p <listen\_on\_port>] [-b <output\_binary\_file>] [-j <output\_json\_file>] [-o <moved\_file\_prefix>]\\
      -i Interface name on which the program will listen. Default is eth0\\
      -p UDP port on which the program will listen. Default is 57843\\
      -b Name of the binary file to where the data is going to be written. Default is miavita.bin\\
      -j Name of the json file to where the data is going to be written. Default is miavita.json\\
      -t Test the program against GPS time. Make sure to compile this program with -D\_\_GPS\_\_.\\
      -o Output file prefix when the file is moved by log rotation. Default is miavita\_old.\\
      -c Buffer capacity expressed in terms of number of packets. Default is 100.
    \end{flushleft}

    Na altura da escrita deste manual, a interface gráfica para o utilizador necessitava de ler os ficheiros json com os pacotes ordenados por \emph{timestamp}. Como tal, o programa que recebe os dados faz \emph{buffering} de X pacotes e antes de os escrever ordena-os. O mecanismo de \emph{buffering} e rotação dos ficheiros encontra-se implementado no ficheiro {\tt list.c}. Embora todos os X pacotes sejam ordenados, apenas $\frac{X}{2}$ são escritos para os ficheiros. Isto serve para evitar ao máximo que um pacote atrasado fique desordenado nos ficheiros.
    
\section{RT2501 - Driver Modificado}\label{sec:driver}

O driver fornecido pela Ralink foi modificado para sincronizar os pacotes de acordo com o algoritmo descrito na minha tese. A principal diferença é que o código actual faz tudo ao nível do driver e não necessita da \emph{framework} de interceptores.

A ideia é conseguir interceptar os pacotes enviados no último momento possível. Neste caso, trata-se da função que submete o pacote ao controlador de USB (função {\tt RTUSBBulkOutDataPacket} no ficheiro {\tt rtusb\_bulk.c}). Por outro lado, também é necessário interceptar os pacotes recebidos o mais cedo possível. Neste caso, trata-se da \emph{callback} chamada pelo controlador de USB (função {\tt REPORT\_ETHERNET\_FRAME\_TO\_LLC} no ficheiro {\tt rtusb\_data.c}).

As funções que sincronizam os pacotes enviados e recebidos encontram-se no ficheiro {\tt sync\_proto.c}. É preciso notar que as funções que submetem pacotes ao controlador de USB e recebem pacotes do mesmo, são chamadas para todos os pacotes enviados e recebidos. Por esta razão é preciso impedir que o \emph{driver} tente sincronizar pacotes que nada têm haver com o módulo \emph{sender\_kthread.ko}. Como por exemplo pacotes ARP. Para este efeito pensou-se no conceito de filtros, muito semelhante ao modo como funcionam as \emph{iptables}. Neste caso, são definidos um conjunto de filtros que indicam qual o tráfego a sincronizar. Estes filtros encontram-se implementados no ficheiro {\tt filter\_chains.c}. Quando nenhum filtro é especificado, nenhum pacote é sincronizado. Contudo, é possível definir filtros para sincronizar, por exemplo, todo o tráfego gerado na porta 57843 pelo IP 192.168.0.123. O driver assume que quando um pacote passa nas especificações de um filtro tem o formato especificado no capitulo \ref{sec:packet_specification}.

Uma vez mais, se todos os outros componentes foram compilados para os testes com GPS, este também o deverá ser (Adicionar a \emph{flag} {\tt EXTRA\_CFLAGS+=-D\_\_GPS\_\_} à Makefile).

Os filtros podem ser definidos em \emph{runtime}. Para tal, o driver implementa uma entrada \emph{proc} ({\tt /proc/synch\_filters}) que recebe dados num dado formato e cria os filtros apropriados. Na directoria {\tt rt2501/sources/Module/user} encontram-se dois programas que tratam de criar e remover filtros. É necessário compilar os programas para o ARM, pelo que a Makefile deve reflectir o compilador apropriado. Opcionalmente pode-se ainda correr o comando {\tt make install} para que os comandos de filtros fiquem instalados no sistema. De modo a saber como usar os comandos, basta corrê-los sem argumentos.

O comando que remove filtros recebe o identificador do filtro a remover. É possível saber este identificador através do comando:

\begin{flushleft}
  {\tt cat /proc/synch\_filters}
\end{flushleft}

Este comando imprime também todas as especificações dos filtros criados. É preciso ter atenção que todos os nós por onde o pacote passa têm de ter os mesmos filtros criados de modo a que o protocolo funcione.

\section{Kernel Modificado}\label{sec:kernel}

O kernel fornecido pela TS foi modificado de modo a conter mais duas \emph{system calls}. Uma é usada para fazer \emph{set} ao tempo do GPS e outro é usada para obter este tempo. O ficheiro {\tt README.rst} no repositório do miavita, explica como adicionar \emph{system calls} ao ARM. Este kernel pode ser obtido fazendo:

\begin{center}
{\tt git clone https://github.com/cnm/ts7500\_kernel}
\end{center}

O tempo do GPS é mantido em duas variáveis dentro do kernel (\_miavita\_elapsed\_secs, \_miavita\_elapsed\_usecs). Uma mantém os segundos dados pelo GPS a outra mantém os microsegundos dados pelo ARM. Não é possível obter menos do que um segundo do GPS, pelo que os microsegundos do ARM são usados. Assume-se que dentro de um segundo o \emph{drift} do cristal do ARM não é significativo.

Ambas as variáveis têm 64 bits e são \emph{signed}. Encontram-se dentro da directoria {\tt ipc} dentro das \emph{sources} do kernel, no ficheiro {\tt miavita\_syscall.c}. A função {\tt pulse\_miavita\_xtime} é chamada pelo módulo \emph{int\_mod.c} (Secção \ref{sec:sender}) a cada interrupção dada pelo PPS do GPS. Isto garante que a variável que mantém os segundos tem a mesma precisão que o PPS do GPS, que é cerca de 50 nanosegundos.

O ficheiro {\tt miavita\_syscall.c} implementa não só as \emph{system calls} criadas, mas também o modo como outros módulos podem interagir com as variáveis que mantém o tempo do GPS. Para que outros módulos usem tais funções é necessário que incluam o ficheiro {\tt miavita\_xtime.h} ({\tt \#include <linux/miavita\_xtime.h>}). Dentro do repositório do miavita, existe uma directoria denominada {\tt interruption/}, onde está um ficheiro com o nome {\tt fpga.c}. Este ficheiro contém exemplos de como estas variáveis podem ser usadas.

\chapter{Especificação do Pacote de Dados}\label{sec:packet_specification}

  \section{Binário}

	Esta secção mostra como os dados são mantidos no ficheiro binário. A figura \ref{fig:packet} mostra a estrutura de cada pacote.
	
	 \begin{figure}[h!]
   	  \centering
   	  \includegraphics[width=0.7\textwidth]{packet.pdf}
	  \caption{Pacote de dados.}
   	  \label{fig:packet}
 	\end{figure}

	Cada campo do pacote é descrito da seguinte forma, quando transmitido na rede:

 \begin{description}
   \item[Time stamp]\hfill\\
   	O primeiro campo do pacote é usado para um valor de tempo, com 64 bits, \emph{signed} e transmitido em \emph{Big Endian}. Este campo representa o tempo de criação do pacote, ou o tempo que passou desde que o pacote foi criado.
   \item[Air time estimation]\hfill\\
   	O segundo campo no pacote é usado para um valor de tempo, com 64 bits, \emph{signed} e transmitido em \emph{Big Endian}. Este campo representa o tempo de transmissão estimado pelo driver.
   \item[Sequence number]\hfill\\
   	O terceiro campo no pacote é usado para um valor de um número de sequência, com 32 bits, \emph{unsigned} e transmitido em \emph{Big Endian}.
   \item[Fails]\hfill\\
         O quarto campo no pacote é usado para fornecer à aplicação o número de falhas que ocorreram desde o último pacote recebido. Este campo pode incluir falhas de todos os pacotes, não só aqueles que pertencem à aplicação.
   \item[Retries]\hfill\\
         O quinto campo no pacote é usado para fornecer à aplicação o número de retransmissões que ocorreram desde o último pacote recebido. Este campo pode incluir retransmissões de todos os pacotes, não só aqueles que pertencem à aplicação.
   \item[Samples]\hfill\\
     O sexto campo é usado para guardar 4 amostras retiradas do Geophone, com 24 bits, \emph{signed} e transmitidas em \emph{Big Endian}.
    \item[Node Identifier]\hfill\\
	O último campo identifica o nó da rede.
 \end{description}

  \section{ASCII}
  
  	Esta secção fornece informação sobre o formato ascii usado para representar os pacotes. O ficheiro de \emph{output} tem um formato JSON, onde as chaves são compostas por {\tt <identificador do nó>:<número de sequência do pacote>}. O seguinte é um exemplo de um ficheiro com o formato descrito.

  \lstinputlisting[language=Python]{packet.py}  
  
  
\chapter{Como Usar os Componentes}

	Esta secção demonstra como correr e instalar cada componente de modo a ter um caso de uso do miavita. Por agora o cartão tem de poder ser escrito. Em primeiro lugar é necessário ter o kernel descrito na secção \ref{sec:kernel} preparado para \emph{flashar} na placa ARM. Este é o procedimento que utilizei:
	
	\begin{enumerate}
		\item No meu computador dentro da pasta que contém o kernel executo {\tt make modules modules\_install zImage}.
		\item Insiro  o cartão do ARM no meu computador.
		\item Apago a pasta {\tt /lib/modules/2.6.24.4/} do cartão, executando\\ {\tt rm -r /media/<particao4>/lib/modules/2.6.24.4/}.
		\item Copio os novos módulos para dentro do cartão, executando\\ {\tt cp -r /lib/modules/2.6.24.4/ /media/<particao4>/lib/modules/}.
		\item Ponho o novo kernel na partição dois do cartão SD, executando\\ {\tt dd if=arch/arm/boot/zImage of=/dev/sdb2} (Atenção que o caminho do \emph{output} pode variar).
		\item Como normalmente quero que o kernel na \emph{flash} seja o mesmo do cartão, faço logo uma cópia para dentro do cartão SD de modo a poder \emph{flashá-lo} mais tarde. Executo \\{\tt dd if=/dev/sdb2 of=/media/<particao4>/root/zImage.dd}
		\item Não esquecer de desmontar o cartão - {\tt umount /media/*}
		\item Ponho o cartão numa placa ARM e em geral uso uma \emph{board} de desenvolvimento para concluir o procedimento.
		\item Ponho a \emph{board} a \emph{bootar} pelo cartão SD.
		\item Em primeiro lugar, faço o \emph{flash} do novo kernel:
			\begin{flushleft}
				{\tt spiflashctl -W 4095 -z 512 -k part1 -i zImage.dd}
			\end{flushleft}
		\item Após concluído, é preciso criar as dependências dos módulos, executando {\tt depmod}.
		\item Por fim, é preciso fazer \emph{reboot} à placa.
	\end{enumerate}

	Em segundo lugar, é preciso pôr o módulo rt73.ko na placa. Este passo pode ser intercalado com o anterior, mas assim fica mais explícito o que é necessário fazer:
	
	\begin{enumerate}
		\item Copio o módulo normalmente para a pasta\\
		 {\tt /lib/modules/2.6.24.4/kernel/drivers/net/wireless/rt2xx0/}.
		\item Após concluído, é preciso criar as dependências dos módulos outra vez, executando {\tt depmod}.
		\item Por fim, é preciso fazer \emph{reboot} à placa.
	\end{enumerate}
	
	Terceiro passo consiste em copiar os programas e módulos para dentro do ARM de modo a poder usá-los.

	\begin{enumerate}
		\item No meu computador, compilo todos os programas e módulos com o \emph{cross-compiler} para o ARM. Isto inclui, programa utilizador que recebe os dados, programas que criam e removem os filtros, o módulo \emph{int\_mod.ko} e o módulo \emph{sender\_kthread.ko}.
		\item Copio todos os componentes para dentro do cartão SD. Em específico, copio os programas {\tt mkfilter} e {\tt rmfilter} para dentro da pasta {\tt /usr/bin/}. De notar que não vale de muito ter o módulo \emph{sender\_kthread.ko} e o programa que recebe os dados na mesma placa.
	\end{enumerate}
	
	\textbf{A partir daqui o sistema está pronto para ser usado. É necessário colocar o cartão em \emph{read-only}}.
	
	Para iniciar um caso de uso do miavita, é necessário matar alguns processos. O seguinte comando deve ser efectuado a cada \emph{boot} da placa:
	
	\begin{center}
		{\tt kill \$(pgrep xuartctl); kill \$(pgrep daqctl);  kill \$(pgrep dioctl); kill -9 \$(pgrep logsave); kill \$(pgrep ts7500ctl); sleep 2; ts7500ctl --autofeed 3; sleep 5; kill \$(pgrep ts7500ctl);}
	\end{center}
	
	De seguida é necessário inserir os módulos para dar início ao programa que recolhe e envia amostras para o sink. De notar que o driver rt2501 modificado já foi inserido automaticamente quando se inseriu a pen wifi.
	
	\begin{enumerate}
		\item Inserir o módulo que recolhe amostras: {\tt insmod int\_mod.ko}
		\item Inserir o módulo que envia as amostras: {\tt insmod  sender\_kthread.ko bind-ip=''<IP da placa wifi>''  sink-ip=''<ip da placa wifi do sink>''}
	\end{enumerate}
	
	Por fim, basta colocar no sink o programa que recebe os dados a correr:
	
	\begin{center}
		{\tt ./main -i rausbwifi}
	\end{center}
	
	A partir deste ponto as placas estarão a enviar dados para o sink e este a recebê-los. Caso seja necessário testar o protocolo com GPS é preciso compilar os componentes com a \emph{flag} {\tt -D\_\_GPS\_\_} e antes de inserir qualquer módulo é preciso inicializar as variáveis no kernel (Secção \ref{sec:kernel}). Para tal é preciso usar o programa {\tt init\_counter} dentro da directoria {\tt kernel\_sender/user/}. Compila-se com:
        
        \begin{center}
          {\tt make CC=arm-unknown-linux-gnu-gcc init\_counter}
        \end{center}
	
        A seguir é preciso correr o programa antes de inserir qualquer módulo. É preciso ter em conta que o programa inicializa o programa {\tt xuartctl}, que tem de ser terminado após o programa {\tt init\_counter} terminar. É ainda preciso ter atenção que o programa tenta abrir o \emph{device} do GPS mais comum que as placas criam ({\tt /dev/pts/1}). É possível especificar outro \emph{device} com a \emph{flag} {\tt -d}. Contudo, o nome do \emph{device} só é conhecido após o programa {\tt xuartctl} ter inicializado. Por esta razão, se o \emph{device} for outro é preciso matar o {\tt xuartctl} antes de correr outra vez o programa {\tt init\_counter}.

\chapter{Framework de Interceptores}

  Dentro da directoria {\tt fred\_framework/} encontra-se o código correspondenete à framework de interceptores. A ideia de tal framework é criar uma estrutura base onde outros módulos do kernel (Interceptores) se possam registar e interceptar o tráfego em cada nó. Por exemple, a agregação de pacotes realizado no projecto miavita é feita através de dois interceptores. Um é responsável por interceptar o tráfego e agregá-lo, enquanto que outro é responsável por desagregá-lo.

  A framework trabalha por cima da API de Netfilters do kernel. Assim que o módulo da framework é inserido no kernel, são registados 5 hooks (Um para cada ponto de intercepção fornecidos pelo Netfilters - Post routing, local in, etc.). Cada interceptor regista-se na framework para poder interceptar o tráfego desejado. Para que cada interceptor só esteja ciente do tráfego que realmente precisa de interceptar, a framework fornece um mecanismo que permite filtrar tráfego não desejado. Isto é atingido através de umas estruturas denominadas \textbf{regras}. As regras são simplesmente um conjunto de \textbf{filtros}. Os filtros são estruturas que possuem um conjunto de especificações e um ponto de interacção. Por exemplo, uma regra para agregar tráfego ao nível da rede e aplicação, com destino à porta 57843 e ip 192.168.0.1 iria criar uma regra com dois filtros. Um filtro iria actuar no ponto de intercepção \emph{local out} e o outro no ponto de intercepção \emph{post routing}. Ambos os filtros iriam conter uma especificação que iria informar a framework que todo o tráfego com aquele destino deve ser interceptado pelo interceptor de agregação.
  
  Em suma, a framework de interceptores permite o registo de um ou mais interceptores que definem regras para que possam interceptar um subconjunto dos pacotes que passam no nó.

%TODO: mkrule e rmrule
  
  \section{Interceptor de agregação} 
  \section{Interceptor de desagregação} 
\end{document}
